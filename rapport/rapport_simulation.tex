\documentclass{report}

\usepackage[latin1]{inputenc}
\usepackage[T1]{fontenc}
\usepackage[francais]{babel}
\usepackage{graphicx}
\usepackage{fancyhdr}

\title{Simulation\\Projet d'examen:\\�tude du caract�re pseudo-al�atoire de \pi}}
\author{Libert Robin BA3 Info\\Umons}
\date{Juin 2018}

\pagestyle{fancy}
\lhead{Libert Robin}
\rhead{BA3 Info}
\cfoot{\thepage}

\begin{document}
\maketitle
\tableofcontents

\chapter{Introduction}


\section{�nonc�}


\begin{center}
		\begin{tabular}{|c|c|c|c|}
				\hline
				100&51.4833&48.8686&54.0128\\
				\hline
				100&52.9167&52.8934&55.0888\\
				\hline
				100&51.5604&51.4268&50.1593\\
				\hline
				100&49.4558&51.5149&49.0166\\
				\hline
				100&49.8615&50.1993&46.7038\\
				\hline
				100&53.6345&52.2414&49.2623\\
				\hline
				100&47.8990&51.3192&49.4790\\
				\hline
				100&52.1283&49.6891&51.2927\\
				\hline
				100&50.1619&50.2381&48.3551\\
				\hline
				100&50.0998&51.5260&56.2141\\
				\hline	
		\end{tabular}
\end{center}

\section{Logiciels utilis�s}
\begin{itemize}
	\item Spyder: ide python
	\item Python 3: pour la r�alisation du code
	\item Numpy (Biblioth�que python)
	\item Mathplotlib: pour la r�alisation des graphiques. (Biblioth�que python)
	\item Windows 10
	\item TeXnicCenter: pour la r�daction du rapport en Latex
\end{itemize}

\chapter{Intuitions}

\section{Premi�re analyse des donn�es}

\begin{center}
	\includegraphics[scale=0.8]{img/compression_audio_100ko_0.png}\\
	Figure 1
\end{center}


\end{document}